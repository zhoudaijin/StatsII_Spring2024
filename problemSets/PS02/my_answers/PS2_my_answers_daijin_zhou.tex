\documentclass[12pt,letterpaper]{article}
\usepackage{graphicx,textcomp}
\usepackage{natbib}
\usepackage{setspace}
\usepackage{fullpage}
\usepackage{color}
\usepackage[reqno]{amsmath}
\usepackage{amsthm}
\usepackage{fancyvrb}
\usepackage{amssymb,enumerate}
\usepackage[all]{xy}
\usepackage{endnotes}
\usepackage{lscape}
\newtheorem{com}{Comment}
\usepackage{float}
\usepackage{hyperref}
\newtheorem{lem} {Lemma}
\newtheorem{prop}{Proposition}
\newtheorem{thm}{Theorem}
\newtheorem{defn}{Definition}
\newtheorem{cor}{Corollary}
\newtheorem{obs}{Observation}
\usepackage[compact]{titlesec}
\usepackage{dcolumn}
\usepackage{tikz}
\usetikzlibrary{arrows}
\usepackage{multirow}
\usepackage{xcolor}
\newcolumntype{.}{D{.}{.}{-1}}
\newcolumntype{d}[1]{D{.}{.}{#1}}
\definecolor{light-gray}{gray}{0.65}
\usepackage{url}
\usepackage{listings}
\usepackage{color}

\definecolor{codegreen}{rgb}{0,0.6,0}
\definecolor{codegray}{rgb}{0.5,0.5,0.5}
\definecolor{codepurple}{rgb}{0.58,0,0.82}
\definecolor{backcolour}{rgb}{0.95,0.95,0.92}

\lstdefinestyle{mystyle}{
	backgroundcolor=\color{backcolour},   
	commentstyle=\color{codegreen},
	keywordstyle=\color{magenta},
	numberstyle=\tiny\color{codegray},
	stringstyle=\color{codepurple},
	basicstyle=\footnotesize,
	breakatwhitespace=false,         
	breaklines=true,                 
	captionpos=b,                    
	keepspaces=true,                 
	numbers=left,                    
	numbersep=5pt,                  
	showspaces=false,                
	showstringspaces=false,
	showtabs=false,                  
	tabsize=2
}
\lstset{style=mystyle}
\newcommand{\Sref}[1]{Section~\ref{#1}}
\newtheorem{hyp}{Hypothesis}

\title{Problem Set 2}
\date{Due: February 18, 2024}
\author{Applied Stats II}


\begin{document}
	\maketitle
	\section*{Instructions}
	\begin{itemize}
		\item Please show your work! You may lose points by simply writing in the answer. If the problem requires you to execute commands in \texttt{R}, please include the code you used to get your answers. Please also include the \texttt{.R} file that contains your code. If you are not sure if work needs to be shown for a particular problem, please ask.
		\item Your homework should be submitted electronically on GitHub in \texttt{.pdf} form.
		\item This problem set is due before 23:59 on Sunday February 18, 2024. No late assignments will be accepted.
	%	\item Total available points for this homework is 80.
	\end{itemize}

	
	%	\vspace{.25cm}
	
%\noindent In this problem set, you will run several regressions and create an add variable plot (see the lecture slides) in \texttt{R} using the \texttt{incumbents\_subset.csv} dataset. Include all of your code.

	\vspace{.25cm}
%\section*{Question 1} %(20 points)}
%\vspace{.25cm}
\noindent We're interested in what types of international environmental agreements or policies people support (\href{https://www.pnas.org/content/110/34/13763}{Bechtel and Scheve 2013)}. So, we asked 8,500 individuals whether they support a given policy, and for each participant, we vary the (1) number of countries that participate in the international agreement and (2) sanctions for not following the agreement. \\

\noindent Load in the data labeled \texttt{climateSupport.RData} on GitHub, which contains an observational study of 8,500 observations.

\begin{itemize}
	\item
	Response variable: 
	\begin{itemize}
		\item \texttt{choice}: 1 if the individual agreed with the policy; 0 if the individual did not support the policy
	\end{itemize}
	\item
	Explanatory variables: 
	\begin{itemize}
		\item
		\texttt{countries}: Number of participating countries [20 of 192; 80 of 192; 160 of 192]
		\item
		\texttt{sanctions}: Sanctions for missing emission reduction targets [None, 5\%, 15\%, and 20\% of the monthly household costs given 2\% GDP growth]
		
	\end{itemize}
	
\end{itemize}

\newpage
\noindent Please answer the following questions:

\begin{enumerate}
	\item
	Remember, we are interested in predicting the likelihood of an individual supporting a policy based on the number of countries participating and the possible sanctions for non-compliance.
	\begin{enumerate}
		\item [] Fit an additive model. Provide the summary output, the global null hypothesis, and $p$-value. Please describe the results and provide a conclusion.\\
		%\item
		%How many iterations did it take to find the maximum likelihood estimates?
		\vspace{.5cm}
		\noindent 
		Interpretation: \\
		Since the dependent variable is binary, the entire question uses a logistic regression model. And use the Likelihood ratio test to test the validity of the coefficients. \\
		Global null hypothesis : all coefficients (coefficients of independent variables) are zero, and there is no significant relationship between independent variables and dependent variable , that is, the independent variables in the model have no predictive power on the dependent variable.
		\begin{table}[h]\begin{center}\begin{tabular}{l c}\hline & Model \\\hline(Intercept)         & $-0.27^{***}$ \\                    & $(0.05)$      \\countries80 of 192  & $0.34^{***}$  \\                    & $(0.05)$      \\countries160 of 192 & $0.65^{***}$  \\                    & $(0.05)$      \\sanctions5\%        & $0.19^{**}$   \\                    & $(0.06)$      \\sanctions15\%       & $-0.13^{*}$   \\                    & $(0.06)$      \\sanctions20\%       & $-0.30^{***}$ \\                    & $(0.06)$      \\\hline AIC                 & $11580.26$    \\BIC                 & $11622.55$    \\Log Likelihood      & $-5784.13$    \\Deviance            & $11568.26$    \\Num. obs.           & $8500$        \\\hline\multicolumn{2}{l}{\scriptsize{$^{***}p<0.001$; $^{**}p<0.01$; $^{*}p<0.05$}}\end{tabular}\caption{Model}\label{table:coefficients}\end{center}\end{table}
		\lstinputlisting[language=R, firstline=41, lastline=95]{PS2_my_answers_daijin_zhou.R}
		\begin{table}[ht]\centering\begin{tabular}{rrrrr}  \hline Resid. Df & Resid. Dev & Df & Deviance & Pr($>$Chi) \\   \hline8499 & 11783.41 &  &  &  \\   8494 & 11568.26 & 5 & 215.15 & 0.0000 \\    \hline\end{tabular}\end{table}
		\noindent
		From the  table, the difference in degrees of freedom is 5, while the difference in bias is 215.15, and the p-value is extremely small (2.2e-16), indicating that the difference in bias between the two models is significant.This suggests that a model that includes the countries and sanctions variables has an advantage in explaining the data relative to the null model.
	\end{enumerate}
	
	\item
	If any of the explanatory variables are significant in this model, then:
	\begin{enumerate}
		\item
		For the policy in which nearly all countries participate [160 of 192], how does increasing sanctions from 5\% to 15\% change the odds that an individual will support the policy? (Interpretation of a coefficient)\\
		\noindent 
		Interpretation: \\		
		Calculate the logarithm of the odds in two different situations, then make the difference, and then exponentialize the difference to get the change in odds ratio.\\
		\lstinputlisting[language=R, firstline=100, lastline=122]{PS2_my_answers_daijin_zhou.R}
		\noindent 
		For the policy in which nearly all countries participate [160 of 192], increasing sanctions from 5\% to 15\% will increase the odds  that an individual will support the policy by 0.7224479.
%		\item
%		For the policy in which very few countries participate [20 of 192], how does increasing sanctions from 5\% to 15\% change the odds that an individual will support the policy? (Interpretation of a coefficient)
		\item
		What is the estimated probability that an individual will support a policy if there are 80 of 192 countries participating with no sanctions? \\
		\lstinputlisting[language=R, firstline=125, lastline=137]{PS2_my_answers_daijin_zhou.R}
		Interpretation: \\		
		The estimated probability that an individual will support a policy if there are 80 of 192 countries participating with no sanctions is about 51.6\%.
		\item
		Would the answers to 2a and 2b potentially change if we included the interaction term in this model? Why? 
		\begin{itemize}
			\item Perform a test to see if including an interaction is appropriate.\\
			\vspace{.5cm}
		Interpretation: \\		
		Use the chi-square test. First fit a logistic regression model containing interaction terms, then use anova() to compare with the original model, and finally analyze the p value in the anova result to see the significance of the coefficient.\\
		\lstinputlisting[language=R, firstline=140, lastline=157]{PS2_my_answers_daijin_zhou.R}
		\begin{table}[ht]\centering\begin{tabular}{rrrrr}  \hline Resid. Df & Resid. Dev & Df & Deviance & Pr($>$Chi) \\   \hline8488 & 11561.97 &  &  &  \\   8494 & 11568.26 & -6 & -6.29 & 0.3912 \\    \hline\end{tabular}\end{table}
		Since the p value is 0.3912 which is bigger than 0.05, it indicates that the deviation difference between the two models is not significant, that is, the interaction term has no significant impact on the explanatory power of the model.The answers to 2a and 2b would not potentially change if we included the interaction term in this model.\\
		
		\end{itemize}
	\end{enumerate}
	\end{enumerate}


\end{document}
