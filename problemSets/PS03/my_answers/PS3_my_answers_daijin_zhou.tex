\documentclass[12pt,letterpaper]{article}
\usepackage{graphicx,textcomp}
\usepackage{natbib}
\usepackage{setspace}
\usepackage{fullpage}
\usepackage{color}
\usepackage[reqno]{amsmath}
\usepackage{amsthm}
\usepackage{fancyvrb}
\usepackage{amssymb,enumerate}
\usepackage[all]{xy}
\usepackage{endnotes}
\usepackage{lscape}
\newtheorem{com}{Comment}
\usepackage{float}
\usepackage{hyperref}
\newtheorem{lem} {Lemma}
\newtheorem{prop}{Proposition}
\newtheorem{thm}{Theorem}
\newtheorem{defn}{Definition}
\newtheorem{cor}{Corollary}
\newtheorem{obs}{Observation}
\usepackage[compact]{titlesec}
\usepackage{dcolumn}
\usepackage{tikz}
\usetikzlibrary{arrows}
\usepackage{multirow}
\usepackage{xcolor}
\newcolumntype{.}{D{.}{.}{-1}}
\newcolumntype{d}[1]{D{.}{.}{#1}}
\definecolor{light-gray}{gray}{0.65}
\usepackage{url}
\usepackage{listings}
\usepackage{color}

\definecolor{codegreen}{rgb}{0,0.6,0}
\definecolor{codegray}{rgb}{0.5,0.5,0.5}
\definecolor{codepurple}{rgb}{0.58,0,0.82}
\definecolor{backcolour}{rgb}{0.95,0.95,0.92}

\lstdefinestyle{mystyle}{
	backgroundcolor=\color{backcolour},   
	commentstyle=\color{codegreen},
	keywordstyle=\color{magenta},
	numberstyle=\tiny\color{codegray},
	stringstyle=\color{codepurple},
	basicstyle=\footnotesize,
	breakatwhitespace=false,         
	breaklines=true,                 
	captionpos=b,                    
	keepspaces=true,                 
	numbers=left,                    
	numbersep=5pt,                  
	showspaces=false,                
	showstringspaces=false,
	showtabs=false,                  
	tabsize=2
}
\lstset{style=mystyle}
\newcommand{\Sref}[1]{Section~\ref{#1}}
\newtheorem{hyp}{Hypothesis}

\title{Problem Set 3}
\date{Due: March 24, 2024}
\author{Applied Stats II}


\begin{document}
	\maketitle
	\section*{Instructions}
	\begin{itemize}
	\item Please show your work! You may lose points by simply writing in the answer. If the problem requires you to execute commands in \texttt{R}, please include the code you used to get your answers. Please also include the \texttt{.R} file that contains your code. If you are not sure if work needs to be shown for a particular problem, please ask.
\item Your homework should be submitted electronically on GitHub in \texttt{.pdf} form.
\item This problem set is due before 23:59 on Sunday March 24, 2024. No late assignments will be accepted.
	\end{itemize}

	\vspace{.25cm}
\section*{Question 1}
\vspace{.25cm}
\noindent We are interested in how governments' management of public resources impacts economic prosperity. Our data come from \href{https://www.researchgate.net/profile/Adam_Przeworski/publication/240357392_Classifying_Political_Regimes/links/0deec532194849aefa000000/Classifying-Political-Regimes.pdf}{Alvarez, Cheibub, Limongi, and Przeworski (1996)} and is labelled \texttt{gdpChange.csv} on GitHub. The dataset covers 135 countries observed between 1950 or the year of independence or the first year forwhich data on economic growth are available ("entry year"), and 1990 or the last year for which data on economic growth are available ("exit year"). The unit of analysis is a particular country during a particular year, for a total $>$ 3,500 observations. 

\begin{itemize}
	\item
	Response variable: 
	\begin{itemize}
		\item \texttt{GDPWdiff}: Difference in GDP between year $t$ and $t-1$. Possible categories include: "positive", "negative", or "no change"
	\end{itemize}
	\item
	Explanatory variables: 
	\begin{itemize}
		\item
		\texttt{REG}: 1=Democracy; 0=Non-Democracy
		\item
		\texttt{OIL}: 1=if the average ratio of fuel exports to total exports in 1984-86 exceeded 50\%; 0= otherwise
	\end{itemize}
	
\end{itemize}
\newpage
\noindent Please answer the following questions:

\begin{enumerate}
	\item Construct and interpret an unordered multinomial logit with \texttt{GDPWdiff} as the output and "no change" as the reference category, including the estimated cutoff points and coefficients.\\
	\lstinputlisting[language=R, firstline=36, lastline=72]{PS3_my_answers_daijin_zhou.R}
	\begin{table}[!htbp] \centering   \caption{Results}   \label{ } \begin{tabular}{@{\extracolsep{5pt}}lD{.}{.}{-3} D{.}{.}{-3} } \\[-1.8ex]\hline \hline \\[-1.8ex]  & \multicolumn{2}{c}{\textit{Dependent variable:}} \\ \cline{2-3} \\[-1.8ex] & \multicolumn{1}{c}{negative} & \multicolumn{1}{c}{positive} \\ \\[-1.8ex] & \multicolumn{1}{c}{(1)} & \multicolumn{1}{c}{(2)}\\ \hline \\[-1.8ex]  REG1 & 1.379^{*} & 1.769^{**} \\   & (0.769) & (0.767) \\   & & \\  OIL1 & 4.784 & 4.576 \\   & (6.885) & (6.885) \\   & & \\  Constant & 3.805^{***} & 4.534^{***} \\   & (0.271) & (0.269) \\   & & \\ \hline \\[-1.8ex] Akaike Inf. Crit. & \multicolumn{1}{c}{4,690.770} & \multicolumn{1}{c}{4,690.770} \\ \hline \hline \\[-1.8ex] \textit{Note:}  & \multicolumn{2}{r}{$^{*}$p$<$0.1; $^{**}$p$<$0.05; $^{***}$p$<$0.01} \\ \end{tabular} \end{table} 
	\begin{table}[htbp]
		\centering
		\caption{Exponentiated Coefficients}
		\begin{tabular}{lccc}
			\hline
			& (Intercept) & REG1 & OIL1 \\
			\hline
			negative & 44.94 & 3.97 & 119.58 \\
			positive & 93.11 & 5.87 & 97.16 \\
			\hline
		\end{tabular}
	\end{table}
	\noindent	
	For negative:\\
	cutoff point is 3.805370;\\
	For every one unit increase in REG, the log-odds of Y = no change vs. Y = negative increase by 1.379282\\
	For every one unit increase in OIL, the log-odds of Y = no change vs. Y = negative increase by 4.783968 \\
	
	For positive:\\
	cutoff point is 4.533759;\\
	For every one unit increase in REG, the log-odds of Y = no change vs. Y = positive increase by 1.769007 \\
	For every one unit increase in OIL, the log-odds of Y = no change vs. Y = positive increase by 4.576321 \\
	
	\item Construct and interpret an ordered multinomial logit with \texttt{GDPWdiff} as the outcome variable, including the estimated cutoff points and coefficients.\\
	\lstinputlisting[language=R, firstline=75, lastline=89]{PS3_my_answers_daijin_zhou.R}
	\begin{table}[!htbp] \centering   \caption{Results}   \label{ } \begin{tabular}{@{\extracolsep{5pt}}lD{.}{.}{-3} } \\[-1.8ex]\hline \hline \\[-1.8ex]  & \multicolumn{1}{c}{\textit{Dependent variable:}} \\ \cline{2-2} \\[-1.8ex] & \multicolumn{1}{c}{GDPWdiff\_category} \\ \hline \\[-1.8ex]  REG1 & 0.410^{***} \\   & (0.075) \\   & \\  OIL1 & -0.179 \\   & (0.115) \\   & \\ \hline \\[-1.8ex] Observations & \multicolumn{1}{c}{3,721} \\ \hline \hline \\[-1.8ex] \textit{Note:}  & \multicolumn{1}{r}{$^{*}$p$<$0.1; $^{**}$p$<$0.05; $^{***}$p$<$0.01} \\ \end{tabular} \end{table} 
\begin{table}[htbp]
	\centering
	\caption{Odds Ratios and 95\% Confidence Intervals}
	\begin{tabular}{lccc}
		\hline
		& OR & 2.5\% & 97.5\% \\
		\hline
		REG1 & 1.51 & 1.30 & 1.75 \\
		OIL1 & 0.84 & 0.67 & 1.05 \\
		\hline
	\end{tabular}
\end{table}
	\noindent	
	Cutoff points are -5.3199 and -0.7036 which are intercepts (I do not know why they disappeared here)\\
	
	if P is less than -5.3199, then predict Yi = no change\\
	if  5.3199 is less than P is less than  -0.7036, then predict Yi = negative\\
	if P is bigger than  -0.7036, then predict Yi is positive;\\
	
	for P(Y is less than or equal to  no change):\\
	For every one unit increase in REG, the log-odds of Y = no change vs. Y = negative increase by 0.4102;\\
	For every one unit increase in OIL, the log-odds of Y is no change vs. Y is negative increase by -0.1788;\\
	
	
	for P(Y is less than or equal to negative):\\
	For every one unit increase in REG, the log-odds of Y = negative vs. Y = positive increase by 0.4102;\\
	For every one unit increase in OIL, the log-odds of Y = negative vs. Y = positive increase by -0.1788;\\
	
	
	
\end{enumerate}

\section*{Question 2} 
\vspace{.25cm}

\noindent Consider the data set \texttt{MexicoMuniData.csv}, which includes municipal-level information from Mexico. The outcome of interest is the number of times the winning PAN presidential candidate in 2006 (\texttt{PAN.visits.06}) visited a district leading up to the 2009 federal elections, which is a count. Our main predictor of interest is whether the district was highly contested, or whether it was not (the PAN or their opponents have electoral security) in the previous federal elections during 2000 (\texttt{competitive.district}), which is binary (1=close/swing district, 0="safe seat"). We also include \texttt{marginality.06} (a measure of poverty) and \texttt{PAN.governor.06} (a dummy for whether the state has a PAN-affiliated governor) as additional control variables. 

\begin{enumerate}
	\item [(a)]
	Run a Poisson regression because the outcome is a count variable. Is there evidence that PAN presidential candidates visit swing districts more? Provide a test statistic and p-value.\\
	\lstinputlisting[language=R, firstline=96, lastline=114]{PS3_my_answers_daijin_zhou.R}
	\begin{table}[!htbp] \centering   \caption{Results}   \label{ } \begin{tabular}{@{\extracolsep{5pt}}lD{.}{.}{-3} } \\[-1.8ex]\hline \hline \\[-1.8ex]  & \multicolumn{1}{c}{\textit{Dependent variable:}} \\ \cline{2-2} \\[-1.8ex] & \multicolumn{1}{c}{PAN.visits.06} \\ \hline \\[-1.8ex]  PAN.governor.061 & -0.312^{*} \\   & (0.167) \\   & \\  competitive.district1 & -0.081 \\   & (0.171) \\   & \\  marginality.06 & -2.080^{***} \\   & (0.117) \\   & \\  Constant & -3.810^{***} \\   & (0.222) \\   & \\ \hline \\[-1.8ex] Observations & \multicolumn{1}{c}{2,407} \\ Log Likelihood & \multicolumn{1}{c}{-645.606} \\ Akaike Inf. Crit. & \multicolumn{1}{c}{1,299.213} \\ \hline \hline \\[-1.8ex] \textit{Note:}  & \multicolumn{1}{r}{$^{*}$p$<$0.1; $^{**}$p$<$0.05; $^{***}$p$<$0.01} \\ \end{tabular} \end{table}
    \noindent	After checking the results of regression, competitive.district’s test statistic is -0.477, P value is 0.6336 which is not significant, so there is no evidence that PAN presidential candidates visit swing districts more.\\
	\item [(b)]
	Interpret the \texttt{marginality.06} and \texttt{PAN.governor.06} coefficients.\\
	\noindent  
	For the coefficient of marginality.06 which is -2.08014 , it means increasing marginality.06 by 1 unit has a multiplicative effect on the mean of Poisson by exp(-2.08014);\\
	
	For the coefficient of PAN.governor.06 which is -0.31158 , it means increasing PAN.governor.06 by 1 unit has a multiplicative effect on the mean of Poisson by exp(-0.31158);\\
	
	\item [(c)]
	Provide the estimated mean number of visits from the winning PAN presidential candidate for a hypothetical district that was competitive (\texttt{competitive.district}=1), had an average poverty level (\texttt{marginality.06} = 0), and a PAN governor (\texttt{PAN.governor.06}=1).
	\lstinputlisting[language=R, firstline=117, lastline=126]{PS3_my_answers_daijin_zhou.R}
	\noindent  
	After calculating, the estimated mean number of visits is 0.01494818, almost 0.
\end{enumerate}

\end{document}
